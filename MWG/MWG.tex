\documentclass{scrartcl}

\usepackage[ngerman]{babel}
\usepackage[utf8]{inputenc}
\usepackage[T1]{fontenc}
\usepackage{graphicx}
\usepackage{amsmath}
\usepackage{chemmacros}
\usepackage{color}
\usepackage{enumitem}
\usepackage{icomma}
\usepackage{titlesec}
\usepackage{tikz}
\usepackage{adjustbox}
\usepackage{multirow}
\usepackage{url}
\usepackage{geometry}
 \geometry{
 a4paper,
 total={170mm,250mm},
 left=20mm,
 top=20mm,
 }

\usepackage[activate={true,nocompatibility},final]{microtype} % better font-rendering
\usepackage[bitstream-charter]{mathdesign} % bitstream font
\titleformat{\section}[hang]{
	\usefont{T1}{bch}{b}{n}\selectfont} % "bch" - Bitstream Character, "b" - bold 
	{} % label
	{0em} % horizontal separation between label and title body
	{\hspace{-0.4pt}\Large \thesection\hspace{0.6em}} % code preceding the title
	[] % additional code following the title body
\titleformat{\subsection}[hang]{
	\usefont{T1}{bch}{b}{n}\selectfont}
	{}
	{0em}
	{\hspace{-0.4pt}\large \thesubsection\hspace{0.6em}}
	[]
\titleformat{\subsubsection}[hang]{
	\usefont{T1}{bch}{b}{n}\selectfont}
	{}
	{0em}
	{\hspace{-0.4pt}\thesubsubsection\hspace{0.6em}}
	[]


\chemsetup{ modules = all }
%\usepackage[version=4]{mhchem}
\chemsetup[redox]{pos=top} % oxid. numbers on top
\usepackage{chemfig}

\newlength{\drop}

\begin{document}
  \begin{titlepage}
    \drop=0.1\textheight
    \centering
    \vspace*{\baselineskip}
    \rule{\textwidth}{1.6pt}\vspace*{-\baselineskip}\vspace*{2pt}
    \rule{\textwidth}{0.4pt}\\[\baselineskip]
    {\LARGE Versuch 1-2 (MWG)\\[0.3\baselineskip] Massenwirkungsgesetz}\\[0.2\baselineskip]
    \rule{\textwidth}{0.4pt}\vspace*{-\baselineskip}\vspace{3.2pt}
    \rule{\textwidth}{1.6pt}\\[\baselineskip]
    \scshape
    {Praktische Einführung in die Chemie\par}
    \vspace*{2\baselineskip}
    \vfill
    {\scshape Versuchtstag:} \        {\large 24.05.2017}\par
  \end{titlepage}
\section{Theorieteil}
\subsection{Grundlagen}\label{subsec:Grundlagen}
Eine der wichtigsten Grundlagen der Chemie ist, dass jede Reaktion prinzipiell in \emph{beide} Richtungen, also von den Edukten zu den Produkten, als auch andersherum, ablaufen kann. Somit gibt es zu jeder Reaktion ein bestimmtes Gleichgewicht, das, wenn eingetreten, gewissermaßen den Stillstand dieses vor- und rückwärts laufenden Prozesses darstellt. 
Dieses Gleichgewicht lässt sich mittels der \emph{Gleichgewichtskonstante} $K_S$ angeben, die aus dem Verhältnis der Produkte jeweils der Konzentrationen der Produkte und der der Edukte hervorgeht. 
Für eine beispielhafte Reaktion 
\begin{subequations}
	\begin{equation}
		a\text{A} + b\text{B} \rightleftharpoons c\text{C} + d\text{D} 
	\end{equation}
	gilt für die Gleichgewichtskonstante:
	\begin{equation}
		K_S = \frac{[\text{C}]^c\cdot[\text{D}]^d}{[\text{A}]^a\cdot[\text{B}]^b}
	\end{equation}
	Wobei die eckigen Klammern für die Konzentration des Stoffes stehen.
\end{subequations}
Liegen die an der Reaktion beiteiligten Stoffe im gaßförmigen Zustand vor, lässt sich anstelle der Konzentration $c$ der \emph{Partialdruck} verwenden, der über das \texttt{ideale Gasgesetz} $p\cdot V = n\cdot R\cdot T$ mit $c = \frac{n}{V} = \frac{p}{R\cdot T}$ angeben. 
\subsection{Freie Reaktionsenthalpie und Temperaturabhängigkeit der Gleichgewichtskonstante}
Die \emph{freie Reaktionsenthalpie} $\Delta_r G$, die mit der \emph{Reaktionsenthalpie} $\Delta H$ im Zusammenhang: $\Delta G = \Delta H - T\Delta S$ steht, gibt Aufschluss darüber, ob eine Reaktion freiwillig oder nur unter Zwang abläuft und lässt sich hier über:
\begin{equation}
	\Delta_r G = \Delta_rG^0 + R\cdot T\cdot\ln{K}
\end{equation}
angeben. Es lassen sich drei Fälle unterscheiden.
\begin{enumerate}
	\item $\Delta G < 0$: Reaktion läuft freiwillig, unter Abgabe von Nutzarbeit ab
	\item $\Delta G = 0$: Das System befindet sich im Gleichgewicht.
	\item $\Delta G > 0$: Reaktion läuft nicht freiwillig, sondern nur unter Zuführung von Energie ab
\end{enumerate}
Für den Fall $\Delta G = 0$ gilt folglich:
\begin{equation}\label{eq:lnK}
	\ln{K} = -\frac{\Delta_r G^0}{R\cdot T} 
\end{equation}
Leitet man nun Gleichung ~\ref{eq:lnK} nach der Termperatur ab, erhält man mit Hilfe der \texttt{Gibbs-Helmholtz-Gleichung}:
	$	\frac{\partial}{\partial T}(\frac{G}{T}) = -\frac{H}{T^2} \label{eq:Gibbs-Helmholtz}$
	die folgende Temperaturabhängigkeit:
	\begin{equation}
		\frac{1}{K}\cdot \frac{dK}{dT} = \frac{d\ln{K}}{dT} = \frac{\Delta_r H^0}{R\cdot T^2} \label{eq:VantHoff}
\end{equation}
Diese Gleichung ~\ref{eq:VantHoff} ist die sogennante \texttt{Van-’t-Hoff-Gleichung}. Nach Integration der selbigen ergibt sich:
\begin{equation}\label{eq:lnK2}
	\ln{K} = -\frac{\Delta_r H^0}{R}\cdot\frac{1}{T} + c
\end{equation}
\subsection{Ammoniaksynthese}
Die Darstellung von Ammoniak durch \ch{H2} und \ch{N2} mit Hilfe eines Katalysators lässt sich mit:
\begin{equation}
	\ch{1/2 N2 + 3/2 H2 <=>  NH3}
\end{equation}
beschreiben. 

Wie in~\ref{subsec:Grundlagen} erwähnt, lässt sich die Gleichgweichtskonstante auch unter Benutzung der Partialdrücke der an Gase angeben (mit $p_0$ als dem Standarddruck):
\begin{equation} \label{eq:Kp}
	K_p = \frac{[\frac{p_{\ch{NH3}}}{p_0}]}{[\frac{p_{\ch{N2}}}{p_0}]^{\frac{1}{2}}\cdot[\frac{p_{\ch{H2}}}{p_0}]^{\frac{3}{2}}}
\end{equation}
Um dieses $K_p$ bestimmen zu können, geht man folgende Überlegung ein: Die Stoffmengen der \emph{ausströmenden} Gase $n_{\ch{H2}}$ und $n_{\ch{N2}}$, die nicht zu \ch{NH3} reagiert sind, lassen sich aus der Differenz der Stoffmenge \emph{eingeströmten} Gase $n_{0,\ch{H2}}$ und $n_{0,\ch{N2}}$ und der des entstandenen $n_{\ch{NH3}}$ bestimmen:
\begin{subequations}
	\begin{align}
		n_{\ch{H2}}=n_{0,\ch{H2}}-\frac{3}{2}n_{\ch{NH3}} \label{eq:nH2} \\
		n_{\ch{N2}}=n_{0,\ch{N2}}-\frac{1}{2}n_{\ch{NH3}} \label{eq:nN2} 
	\end{align}
\end{subequations}
Die Summe der Stoffmengen der ausströmenden Gase muss der Gesamtstoffmenge entsprechen, also $n_{ges} = n_{\ch{H2}} + n_{\ch{N2}} + n_{\ch{NH3}} = n_{0,\ch{H2}} + n_{0,\ch{N2}} - n_{\ch{NH3}}$. \\
Da nun $n_{\ch{NH3}}<<n_{0,\ch{H2}},n_{0,\ch{N2}}$ in diesem Versuch, kann für die Gesamtstoffmenge:
\begin{equation} \label{eq:nges}
	n_{ges} \cong n_{0,\ch{H2}} + n_{0,\ch{N2}} 
\end{equation}
in guter Näherung angenommen werden. Unter Ausnutzung des \texttt{idealen Gasgesetztes} $p\cdot V = n\cdot R\cdot T$ und der Strömungsgeschwindigkeit $\dot{V}=\frac{V}{t}$, die den Volumenstrom angibt, können die Stoffmengen wie folgt ausgedrückt werden:
\begin{subequations}
	\begin{align}
		n_{0,\ch{H2}} &= \frac{p_0\cdot \dot{V}_{\ch{H2}}\cdot t}{R\cdot T} \\
		n_{0,\ch{N2}} &= \frac{p_0\cdot \dot{V}_{\ch{N2}}\cdot t}{R\cdot T}
	\end{align}
	Woraus sich mit Gleichung~\ref{eq:nges}
	\begin{align}
		n_{ges} &= \frac{p_0\cdot(\dot{V}_{\ch{H2}}+\dot{V}_{\ch{N2}})\cdot t}{R\cdot T} \label{eq:nges2}
	\end{align}
		ergibt.
\end{subequations}
Mit dem \texttt{Daltonschem Gesetz} ist der Partialdruck eines Stoffes das Verhältnis der Stoffmenge dieses Stoffes zur Gesamtstoffmenge multipliziert mit dem Gesamtdruck. Also hier:
\begin{subequations}
	\begin{align}
		p_{\ch{H2}} = \frac{n_{\ch{H2}}}{n_{ges}}\cdot p &= \frac{\dot{V}_{\ch{H2}}}{\dot{V}_{\ch{N2}} + \dot{V}_{\ch{H2}}}\cdot p \label{eq:pH2} \\
		p_{\ch{N2}} = \frac{n_{\ch{N2}}}{n_{ges}}\cdot p &= \frac{\dot{V}_{\ch{N2}}}{\dot{V}_{\ch{N2}} + \dot{V}_{\ch{H2}}}\cdot p \label{eq:pN2} \\
		p_{\ch{NH3}} &= \frac{n_{\ch{NH3}}}{n_{ges}}\cdot p \label{eq:pnh3}
	\end{align}
	 Dabei folgt aus Gleichung~\ref{eq:nges2} und~\ref{eq:pnh3}
	\begin{align}
		p_{\ch{NH3}} &= n_{\ch{NH3}}\cdot\frac{R\cdot T}{p_0\cdot(\dot{V}_{\ch{N2}}+\dot{V}_{\ch{H2}})}\cdot p
	\end{align}
\end{subequations}
Gleichungen~\ref{eq:pH2},~\ref{eq:pN2} und~\ref{eq:pnh3} in Gleichung~\ref{eq:Kp} eingesetz, ergibt:
\begin{equation}
	K_p = \frac{n_{\ch{NH3}}\cdot\frac{R\cdot T}{p_0\cdot(\dot{V}_{\ch{N2}}+\cdot{V}_{\ch{H2}}}\cdot p}{[\frac{\dot{V}_{\ch{H2}}}{\dot{V}_{\ch{N2}} + \dot{V}_{\ch{H2}}}\cdot p]^{\frac{3}{2}}\cdot [\frac{\dot{V}_{\ch{N2}}}{\dot{V}_{\ch{N2}} + \dot{V}_{\ch{H2}}}\cdot p]^{\frac{1}{2}}} = \frac{24,377\frac{\text{l}}{\text{mol}}\cdot n_{\ch{NH3}}\cdot(\dot{V}_{\ch{H2}} + \dot{V}_{\ch{N2}})}{\dot{V}_{\ch{N2}}^{\frac{1}{2}}\cdot\dot{V}_{\ch{H2}}^{\frac{3}{2}}\cdot t\cdot p}\cdot p_0
\end{equation}
\section{Versuche}
\subsection{Bestimmung der Gleichgewichtskonstanten \emph{K\textsubscript{\emph{p}}} der Ammoniaksynthese in Abhängigkeit der Temperatur}
\subsubsection{Aufgabenstellung}
Es sollen jeweils 4 Zeitmessungen bei 4 verschiedenen Temperaturen (zwischen 500$^\circ$C und 700$^\circ$C) genommen werden mit unterschiedlichen Gasgeschwindigkeiten. Daraus soll dann die Gleichgewichtskonstante \emph{K\textsubscript{\emph{p}}} und die Standardbildungsenthalpie bestimmt werden.
\subsubsection{Versuchsaufbau}
\begin{figure}[h]
  \centering
     \includegraphics[width=1.0\textwidth]{Versuchsaufbau.png}
  \caption{Schematischer Aufbau der Apparatur zur Ammoniaksynthese. Quelle: Skript S. 28}
  \label{fig:Bild1}
\end{figure}
\subsubsection{Versuchsdurchführung}
Zu erst wurde der Ofen auf 500$^\circ$C vorgeheizt und die Zuleitung für Stickstoff und Wasserstoff geöffnet sowie der Umgebungsdruck abgelesen. Anschließend wurden zu 50 ml einer 5$\cdot10$\textsuperscript{-4} N (2,5$\cdot10$\textsuperscript{-4} M) \ch{H2SO4}-Lösung 5 Tropfen einer Methylrot-Lösung zugegeben(im Erlenmyerkolben). Danach wurde so lange eine NAOH-Lösung dazu gegeben bis ein Umschlag von rot nach zitronengelb stattgefunden hat. Dies dient später beim eigentlichen Versuch zum Vergleich. Bei dem eigentlichen Versuch wurden 4 verschiedene Temperaturen gewählt (hier: 500,570,640 und 700$^\circ$C). Zusätzlich wurde jedes mal ein Erlenmeyerkolben bereitgestellt, gemischt mit 50 ml Schwefelsäure und 5 Tropfen Methylrot plus einen Rührfisch. Diese Mischung wurde beim Gaseinleitungsrohr befestigt durch welches später das entstandene Gas in die Lösung gebracht wurde. Bei jeder Temperatur wurden jeweils 4 Messungen gemacht in welchen jedes mal die Strömungsgeschwindigkeit verändert wurde. Sobald der 3-Wege-Hahn zum Erlenmeyerkolben geöffnet wurde, wurde die Zeit genommen bis ein Farbumschlag sichtbar wurde.
\begin{figure}
	\centering
	\caption{Messwerte}
	\begin{tabular}{c l r r r}
		$T$ in K & Messung & $\dot{V}_{\ch{N2}}$ in $[\frac{\text{Nl}}{\text{h}}]$ & $\dot{V}_{\ch{H2}}$ in $[\frac{\text{Nl}}{\text{h}}]$ & $t$ in [s] \\ \hline \hline
		\multirow{4}{*}{773,15} & 1 & 3,707 & 11,361 & 111 \\
		& 2 & 10,325 & 30,455 & 41 \\
		& 3 & 15,518 & 45,174 & 34 \\
		& 4 & 22,073 & 65,948 & 25 \\ \hline
		\multirow{4}{*}{843,15} & 1 & 3,707 & 11,361 & 156 \\
		& 2 & 10,325 & 30,455 & 68 \\
		& 3 & 15,518 & 45,174 & 47 \\
		& 4 & 22,073 & 65,948 & 33 \\ \hline
		\multirow{4}{*}{913,15} & 1 & 3,707 & 11,361 & 257 \\
		& 2 & 10,325 & 30,455 & 113 \\
		& 3 & 15,518 & 45,174 & 82 \\
		& 4 & 22,073 & 65,948 & 59 \\ \hline
		\multirow{4}{*}{973,15} & 1 & 3,707 & 11,361 & 420 \\
		& 2 & 10,325 & 30,455 & 189 \\
		& 3 & 15,518 & 45,174 & 130 \\
		& 4 & 22,073 & 65,948 & 98 
	\end{tabular}
\end{figure}

\subsubsection{Auswertung}
\textbf{Reaktionsgleichung:}
\ch{1/2 N2 + 3/2 H2 <=> NH3} \\
\textbf{Messwerte:}
Berechnung der Gleichgewichtskonstanten \emph{K\textsubscript{\emph{p}}} mit\\ \\
\begin{equation}
K_p = \frac{ R \cdot T_0 \cdot p_0 \cdot n_{NH\textsubscript{3}} \cdot ( \dot{V}_{N\textsubscript{2}} + \dot{V}_{H\textsubscript{2}})}{\dot{V}^{1/2}_{N\textsubscript{2}} + \dot{V}^{3/2}_{H\textsubscript{2}} \cdot t \cdot p}
\end{equation}
ergibt für die erste Messung:
\begin{equation}
	K_p = \frac{24,337 \;\frac{\text{l} \cdot \text{bar}}{\text{mol}} \cdot 2,5 \cdot 10^{-5} \;\text{mol} \cdot ((\frac{3,707}{3600}) + (\frac{11,361}{3600}))\frac{\text{Nl}}{\text{s}}}{(\frac{3,707}{3600}\frac{\text{Nl}}{\text{s}})^{1/2} + (\frac{11,361}{3600}\frac{\text{Nl}}{\text{s}})^{3/2} \cdot 111 \;\text{s} \cdot 0,968 \;\text{bar}} = 0,00215 = 2,15 \cdot 10^{-3}.
\end{equation}
Es wurde mit R = 0,08319 $\frac{\text{l}\cdot \text{bar}}{\text{mol}\cdot \text{K}}$, T\textsubscript{0} = 293,15 K, p\textsubscript{0} = 1 bar und dem gemessenen Umgebungsdruck von $p = 1,018\; \text{bar} - 0,05 \text{ bar} = 0,968 \text{ bar}$ gerechnet.
\begin{figure}[h]
	\centering
	\caption{Gleichgewichtskonstante $K_p$ und $\ln{K_p}$}
	\begin{tabular}{c|c c|c c|c c|c c}
		$T$ in K & \multicolumn{2}{|c|}{Messung 1} & \multicolumn{2}{c|}{Messung 2} & \multicolumn{2}{c|}{Messung 3} & \multicolumn{2}{c}{Messung 4} \\ \hline \hline
		& $K_p$ & $\ln{K_p}$ & $K_p$ & $\ln{K_p}$ & $K_p$ & $\ln{K_p}$ & $K_p$ & $\ln{K_p}$ \\ \hline
		773,15 & $4,173\cdot 10^{-3}$ & -5,479 & $4,174\cdot 10^{-3}$ &-5,479 & $3,383\cdot 10^{-3}$ & -5,689 & $3,171\cdot 10^{-3}$ & -5,754 \\
		843,15 & $2,969\cdot 10^{-3}$ &-5,819 & $2,517\cdot 10^{-3}$ & -5,985 & $2,447\cdot 10^{-3}$ & -6,013 & $2,403\cdot 10^{-3}$ & -6,031 \\
		913,15 & $1,802\cdot 10^{-3}$ & -6,319 & $1,515\cdot 10^{-3}$ & -6,493 & $1,403\cdot 10^{-3}$ & -6,569 & $1,344\cdot 10^{-3}$ & -6,612 \\
		973,15 & $1,103\cdot 10^{-3}$ & -6,810 & $9,055\cdot 10^{-4}$ & -7,007 & $8,847\cdot 10^{-4}$ & -7,03 & $8,090\cdot 10^{-4}$ & -7,120
	\end{tabular}
\end{figure}

\begin{figure}\label{fig:plot}
	\centering
	\caption{$\ln{K_p}$ über $\frac{1}{T}$}
	% GNUPLOT: LaTeX picture
\setlength{\unitlength}{0.240900pt}
\ifx\plotpoint\undefined\newsavebox{\plotpoint}\fi
\sbox{\plotpoint}{\rule[-0.200pt]{0.400pt}{0.400pt}}%
\begin{picture}(1500,900)(0,0)
\sbox{\plotpoint}{\rule[-0.200pt]{0.400pt}{0.400pt}}%
\put(211.0,131.0){\rule[-0.200pt]{4.818pt}{0.400pt}}
\put(191,131){\makebox(0,0)[r]{$0$}}
\put(1419.0,131.0){\rule[-0.200pt]{4.818pt}{0.400pt}}
\put(211.0,222.0){\rule[-0.200pt]{4.818pt}{0.400pt}}
\put(191,222){\makebox(0,0)[r]{$0.2$}}
\put(1419.0,222.0){\rule[-0.200pt]{4.818pt}{0.400pt}}
\put(211.0,313.0){\rule[-0.200pt]{4.818pt}{0.400pt}}
\put(191,313){\makebox(0,0)[r]{$0.4$}}
\put(1419.0,313.0){\rule[-0.200pt]{4.818pt}{0.400pt}}
\put(211.0,404.0){\rule[-0.200pt]{4.818pt}{0.400pt}}
\put(191,404){\makebox(0,0)[r]{$0.6$}}
\put(1419.0,404.0){\rule[-0.200pt]{4.818pt}{0.400pt}}
\put(211.0,495.0){\rule[-0.200pt]{4.818pt}{0.400pt}}
\put(191,495){\makebox(0,0)[r]{$0.8$}}
\put(1419.0,495.0){\rule[-0.200pt]{4.818pt}{0.400pt}}
\put(211.0,586.0){\rule[-0.200pt]{4.818pt}{0.400pt}}
\put(191,586){\makebox(0,0)[r]{$1$}}
\put(1419.0,586.0){\rule[-0.200pt]{4.818pt}{0.400pt}}
\put(211.0,677.0){\rule[-0.200pt]{4.818pt}{0.400pt}}
\put(191,677){\makebox(0,0)[r]{$1.2$}}
\put(1419.0,677.0){\rule[-0.200pt]{4.818pt}{0.400pt}}
\put(211.0,768.0){\rule[-0.200pt]{4.818pt}{0.400pt}}
\put(191,768){\makebox(0,0)[r]{$1.4$}}
\put(1419.0,768.0){\rule[-0.200pt]{4.818pt}{0.400pt}}
\put(211.0,859.0){\rule[-0.200pt]{4.818pt}{0.400pt}}
\put(191,859){\makebox(0,0)[r]{$1.6$}}
\put(1419.0,859.0){\rule[-0.200pt]{4.818pt}{0.400pt}}
\put(211.0,131.0){\rule[-0.200pt]{0.400pt}{4.818pt}}
\put(211,90){\makebox(0,0){$300$}}
\put(211.0,839.0){\rule[-0.200pt]{0.400pt}{4.818pt}}
\put(334.0,131.0){\rule[-0.200pt]{0.400pt}{4.818pt}}
\put(334,90){\makebox(0,0){$320$}}
\put(334.0,839.0){\rule[-0.200pt]{0.400pt}{4.818pt}}
\put(457.0,131.0){\rule[-0.200pt]{0.400pt}{4.818pt}}
\put(457,90){\makebox(0,0){$340$}}
\put(457.0,839.0){\rule[-0.200pt]{0.400pt}{4.818pt}}
\put(579.0,131.0){\rule[-0.200pt]{0.400pt}{4.818pt}}
\put(579,90){\makebox(0,0){$360$}}
\put(579.0,839.0){\rule[-0.200pt]{0.400pt}{4.818pt}}
\put(702.0,131.0){\rule[-0.200pt]{0.400pt}{4.818pt}}
\put(702,90){\makebox(0,0){$380$}}
\put(702.0,839.0){\rule[-0.200pt]{0.400pt}{4.818pt}}
\put(825.0,131.0){\rule[-0.200pt]{0.400pt}{4.818pt}}
\put(825,90){\makebox(0,0){$400$}}
\put(825.0,839.0){\rule[-0.200pt]{0.400pt}{4.818pt}}
\put(948.0,131.0){\rule[-0.200pt]{0.400pt}{4.818pt}}
\put(948,90){\makebox(0,0){$420$}}
\put(948.0,839.0){\rule[-0.200pt]{0.400pt}{4.818pt}}
\put(1071.0,131.0){\rule[-0.200pt]{0.400pt}{4.818pt}}
\put(1071,90){\makebox(0,0){$440$}}
\put(1071.0,839.0){\rule[-0.200pt]{0.400pt}{4.818pt}}
\put(1193.0,131.0){\rule[-0.200pt]{0.400pt}{4.818pt}}
\put(1193,90){\makebox(0,0){$460$}}
\put(1193.0,839.0){\rule[-0.200pt]{0.400pt}{4.818pt}}
\put(1316.0,131.0){\rule[-0.200pt]{0.400pt}{4.818pt}}
\put(1316,90){\makebox(0,0){$480$}}
\put(1316.0,839.0){\rule[-0.200pt]{0.400pt}{4.818pt}}
\put(1439.0,131.0){\rule[-0.200pt]{0.400pt}{4.818pt}}
\put(1439,90){\makebox(0,0){$500$}}
\put(1439.0,839.0){\rule[-0.200pt]{0.400pt}{4.818pt}}
\put(211.0,131.0){\rule[-0.200pt]{0.400pt}{175.375pt}}
\put(211.0,131.0){\rule[-0.200pt]{295.825pt}{0.400pt}}
\put(1439.0,131.0){\rule[-0.200pt]{0.400pt}{175.375pt}}
\put(211.0,859.0){\rule[-0.200pt]{295.825pt}{0.400pt}}
\put(30,495){\makebox(0,0){Intensität}}
\put(825,29){\makebox(0,0){Wellenlänge in nm}}
\put(1279,818){\makebox(0,0)[r]{neutral}}
\put(1299.0,818.0){\rule[-0.200pt]{24.090pt}{0.400pt}}
\put(272,303){\usebox{\plotpoint}}
\multiput(272.58,303.00)(0.499,0.613){121}{\rule{0.120pt}{0.590pt}}
\multiput(271.17,303.00)(62.000,74.775){2}{\rule{0.400pt}{0.295pt}}
\multiput(334.00,379.58)(0.516,0.499){115}{\rule{0.514pt}{0.120pt}}
\multiput(334.00,378.17)(59.934,59.000){2}{\rule{0.257pt}{0.400pt}}
\multiput(395.58,438.00)(0.499,0.548){121}{\rule{0.120pt}{0.539pt}}
\multiput(394.17,438.00)(62.000,66.882){2}{\rule{0.400pt}{0.269pt}}
\multiput(457.00,506.58)(0.516,0.499){115}{\rule{0.514pt}{0.120pt}}
\multiput(457.00,505.17)(59.934,59.000){2}{\rule{0.257pt}{0.400pt}}
\multiput(518.00,565.58)(2.223,0.494){25}{\rule{1.843pt}{0.119pt}}
\multiput(518.00,564.17)(57.175,14.000){2}{\rule{0.921pt}{0.400pt}}
\multiput(579.00,577.92)(0.915,-0.498){65}{\rule{0.829pt}{0.120pt}}
\multiput(579.00,578.17)(60.279,-34.000){2}{\rule{0.415pt}{0.400pt}}
\multiput(641.00,543.92)(0.649,-0.498){91}{\rule{0.619pt}{0.120pt}}
\multiput(641.00,544.17)(59.715,-47.000){2}{\rule{0.310pt}{0.400pt}}
\multiput(702.00,496.92)(0.943,-0.497){63}{\rule{0.852pt}{0.120pt}}
\multiput(702.00,497.17)(60.233,-33.000){2}{\rule{0.426pt}{0.400pt}}
\multiput(764.00,463.92)(2.070,-0.494){27}{\rule{1.727pt}{0.119pt}}
\multiput(764.00,464.17)(57.416,-15.000){2}{\rule{0.863pt}{0.400pt}}
\multiput(825.00,448.92)(0.957,-0.497){61}{\rule{0.863pt}{0.120pt}}
\multiput(825.00,449.17)(59.210,-32.000){2}{\rule{0.431pt}{0.400pt}}
\multiput(886.00,416.92)(0.608,-0.498){99}{\rule{0.586pt}{0.120pt}}
\multiput(886.00,417.17)(60.783,-51.000){2}{\rule{0.293pt}{0.400pt}}
\multiput(948.00,365.92)(0.525,-0.499){113}{\rule{0.521pt}{0.120pt}}
\multiput(948.00,366.17)(59.919,-58.000){2}{\rule{0.260pt}{0.400pt}}
\multiput(1009.00,307.92)(0.574,-0.498){105}{\rule{0.559pt}{0.120pt}}
\multiput(1009.00,308.17)(60.839,-54.000){2}{\rule{0.280pt}{0.400pt}}
\multiput(1071.00,253.92)(0.745,-0.498){79}{\rule{0.695pt}{0.120pt}}
\multiput(1071.00,254.17)(59.557,-41.000){2}{\rule{0.348pt}{0.400pt}}
\multiput(1132.00,212.92)(1.229,-0.497){47}{\rule{1.076pt}{0.120pt}}
\multiput(1132.00,213.17)(58.767,-25.000){2}{\rule{0.538pt}{0.400pt}}
\multiput(1193.00,187.92)(2.105,-0.494){27}{\rule{1.753pt}{0.119pt}}
\multiput(1193.00,188.17)(58.361,-15.000){2}{\rule{0.877pt}{0.400pt}}
\multiput(1255.00,172.93)(4.612,-0.485){11}{\rule{3.586pt}{0.117pt}}
\multiput(1255.00,173.17)(53.558,-7.000){2}{\rule{1.793pt}{0.400pt}}
\put(1316,165.17){\rule{12.500pt}{0.400pt}}
\multiput(1316.00,166.17)(36.056,-2.000){2}{\rule{6.250pt}{0.400pt}}
\put(1378,163.67){\rule{14.695pt}{0.400pt}}
\multiput(1378.00,164.17)(30.500,-1.000){2}{\rule{7.347pt}{0.400pt}}
\put(272,303){\makebox(0,0){$+$}}
\put(334,379){\makebox(0,0){$+$}}
\put(395,438){\makebox(0,0){$+$}}
\put(457,506){\makebox(0,0){$+$}}
\put(518,565){\makebox(0,0){$+$}}
\put(579,579){\makebox(0,0){$+$}}
\put(641,545){\makebox(0,0){$+$}}
\put(702,498){\makebox(0,0){$+$}}
\put(764,465){\makebox(0,0){$+$}}
\put(825,450){\makebox(0,0){$+$}}
\put(886,418){\makebox(0,0){$+$}}
\put(948,367){\makebox(0,0){$+$}}
\put(1009,309){\makebox(0,0){$+$}}
\put(1071,255){\makebox(0,0){$+$}}
\put(1132,214){\makebox(0,0){$+$}}
\put(1193,189){\makebox(0,0){$+$}}
\put(1255,174){\makebox(0,0){$+$}}
\put(1316,167){\makebox(0,0){$+$}}
\put(1378,165){\makebox(0,0){$+$}}
\put(1439,164){\makebox(0,0){$+$}}
\put(1349,818){\makebox(0,0){$+$}}
\put(1279,777){\makebox(0,0)[r]{leicht sauer}}
\multiput(1299,777)(20.756,0.000){5}{\usebox{\plotpoint}}
\put(1399,777){\usebox{\plotpoint}}
\put(272,417){\usebox{\plotpoint}}
\multiput(272,417)(20.491,3.305){4}{\usebox{\plotpoint}}
\multiput(334,427)(19.314,-7.599){3}{\usebox{\plotpoint}}
\multiput(395,403)(19.845,-6.081){3}{\usebox{\plotpoint}}
\multiput(457,384)(19.816,-6.172){3}{\usebox{\plotpoint}}
\multiput(518,365)(18.625,-9.160){3}{\usebox{\plotpoint}}
\multiput(579,335)(17.823,-10.636){4}{\usebox{\plotpoint}}
\multiput(641,298)(18.503,-9.403){3}{\usebox{\plotpoint}}
\multiput(702,267)(19.932,-5.787){3}{\usebox{\plotpoint}}
\multiput(764,249)(20.533,-3.029){3}{\usebox{\plotpoint}}
\multiput(825,240)(20.482,-3.358){3}{\usebox{\plotpoint}}
\multiput(886,230)(20.377,-3.944){3}{\usebox{\plotpoint}}
\multiput(948,218)(20.155,-4.956){3}{\usebox{\plotpoint}}
\multiput(1009,203)(20.377,-3.944){3}{\usebox{\plotpoint}}
\multiput(1071,191)(20.482,-3.358){3}{\usebox{\plotpoint}}
\multiput(1132,181)(20.656,-2.032){3}{\usebox{\plotpoint}}
\multiput(1193,175)(20.731,-1.003){3}{\usebox{\plotpoint}}
\multiput(1255,172)(20.753,-0.340){3}{\usebox{\plotpoint}}
\multiput(1316,171)(20.756,0.000){3}{\usebox{\plotpoint}}
\multiput(1378,171)(20.753,0.340){3}{\usebox{\plotpoint}}
\put(1439,172){\usebox{\plotpoint}}
\put(272,417){\makebox(0,0){$\times$}}
\put(334,427){\makebox(0,0){$\times$}}
\put(395,403){\makebox(0,0){$\times$}}
\put(457,384){\makebox(0,0){$\times$}}
\put(518,365){\makebox(0,0){$\times$}}
\put(579,335){\makebox(0,0){$\times$}}
\put(641,298){\makebox(0,0){$\times$}}
\put(702,267){\makebox(0,0){$\times$}}
\put(764,249){\makebox(0,0){$\times$}}
\put(825,240){\makebox(0,0){$\times$}}
\put(886,230){\makebox(0,0){$\times$}}
\put(948,218){\makebox(0,0){$\times$}}
\put(1009,203){\makebox(0,0){$\times$}}
\put(1071,191){\makebox(0,0){$\times$}}
\put(1132,181){\makebox(0,0){$\times$}}
\put(1193,175){\makebox(0,0){$\times$}}
\put(1255,172){\makebox(0,0){$\times$}}
\put(1316,171){\makebox(0,0){$\times$}}
\put(1378,171){\makebox(0,0){$\times$}}
\put(1439,172){\makebox(0,0){$\times$}}
\put(1349,777){\makebox(0,0){$\times$}}
\sbox{\plotpoint}{\rule[-0.400pt]{0.800pt}{0.800pt}}%
\sbox{\plotpoint}{\rule[-0.200pt]{0.400pt}{0.400pt}}%
\put(1279,736){\makebox(0,0)[r]{stark alkalisch}}
\sbox{\plotpoint}{\rule[-0.400pt]{0.800pt}{0.800pt}}%
\put(1299.0,736.0){\rule[-0.400pt]{24.090pt}{0.800pt}}
\put(272,250){\usebox{\plotpoint}}
\multiput(273.41,250.00)(0.502,1.013){117}{\rule{0.121pt}{1.813pt}}
\multiput(270.34,250.00)(62.000,121.237){2}{\rule{0.800pt}{0.906pt}}
\multiput(335.41,375.00)(0.502,1.005){115}{\rule{0.121pt}{1.800pt}}
\multiput(332.34,375.00)(61.000,118.264){2}{\rule{0.800pt}{0.900pt}}
\multiput(396.41,497.00)(0.502,1.094){117}{\rule{0.121pt}{1.942pt}}
\multiput(393.34,497.00)(62.000,130.969){2}{\rule{0.800pt}{0.971pt}}
\multiput(458.41,632.00)(0.502,0.988){115}{\rule{0.121pt}{1.774pt}}
\multiput(455.34,632.00)(61.000,116.318){2}{\rule{0.800pt}{0.887pt}}
\multiput(518.00,753.41)(0.679,0.502){83}{\rule{1.284pt}{0.121pt}}
\multiput(518.00,750.34)(58.334,45.000){2}{\rule{0.642pt}{0.800pt}}
\multiput(579.00,795.09)(0.892,-0.503){63}{\rule{1.617pt}{0.121pt}}
\multiput(579.00,795.34)(58.644,-35.000){2}{\rule{0.809pt}{0.800pt}}
\multiput(641.00,760.09)(0.499,-0.502){115}{\rule{1.000pt}{0.121pt}}
\multiput(641.00,760.34)(58.924,-61.000){2}{\rule{0.500pt}{0.800pt}}
\multiput(702.00,699.09)(0.563,-0.502){103}{\rule{1.102pt}{0.121pt}}
\multiput(702.00,699.34)(59.713,-55.000){2}{\rule{0.551pt}{0.800pt}}
\multiput(764.00,644.09)(1.856,-0.507){27}{\rule{3.071pt}{0.122pt}}
\multiput(764.00,644.34)(54.627,-17.000){2}{\rule{1.535pt}{0.800pt}}
\multiput(825.00,627.09)(0.664,-0.502){85}{\rule{1.261pt}{0.121pt}}
\multiput(825.00,627.34)(58.383,-46.000){2}{\rule{0.630pt}{0.800pt}}
\multiput(887.41,577.94)(0.502,-0.637){117}{\rule{0.121pt}{1.219pt}}
\multiput(884.34,580.47)(62.000,-76.469){2}{\rule{0.800pt}{0.610pt}}
\multiput(949.41,498.11)(0.502,-0.764){115}{\rule{0.121pt}{1.420pt}}
\multiput(946.34,501.05)(61.000,-90.053){2}{\rule{0.800pt}{0.710pt}}
\multiput(1010.41,405.62)(0.502,-0.686){117}{\rule{0.121pt}{1.297pt}}
\multiput(1007.34,408.31)(62.000,-82.308){2}{\rule{0.800pt}{0.648pt}}
\multiput(1072.41,321.58)(0.502,-0.540){115}{\rule{0.121pt}{1.066pt}}
\multiput(1069.34,323.79)(61.000,-63.788){2}{\rule{0.800pt}{0.533pt}}
\multiput(1132.00,258.09)(0.746,-0.502){75}{\rule{1.390pt}{0.121pt}}
\multiput(1132.00,258.34)(58.114,-41.000){2}{\rule{0.695pt}{0.800pt}}
\multiput(1193.00,217.09)(1.315,-0.504){41}{\rule{2.267pt}{0.122pt}}
\multiput(1193.00,217.34)(57.295,-24.000){2}{\rule{1.133pt}{0.800pt}}
\multiput(1255.00,193.08)(2.982,-0.512){15}{\rule{4.636pt}{0.123pt}}
\multiput(1255.00,193.34)(51.377,-11.000){2}{\rule{2.318pt}{0.800pt}}
\multiput(1316.00,182.06)(9.995,-0.560){3}{\rule{10.120pt}{0.135pt}}
\multiput(1316.00,182.34)(40.995,-5.000){2}{\rule{5.060pt}{0.800pt}}
\put(1378,176.84){\rule{14.695pt}{0.800pt}}
\multiput(1378.00,177.34)(30.500,-1.000){2}{\rule{7.347pt}{0.800pt}}
\put(272,250){\makebox(0,0){$\ast$}}
\put(334,375){\makebox(0,0){$\ast$}}
\put(395,497){\makebox(0,0){$\ast$}}
\put(457,632){\makebox(0,0){$\ast$}}
\put(518,752){\makebox(0,0){$\ast$}}
\put(579,797){\makebox(0,0){$\ast$}}
\put(641,762){\makebox(0,0){$\ast$}}
\put(702,701){\makebox(0,0){$\ast$}}
\put(764,646){\makebox(0,0){$\ast$}}
\put(825,629){\makebox(0,0){$\ast$}}
\put(886,583){\makebox(0,0){$\ast$}}
\put(948,504){\makebox(0,0){$\ast$}}
\put(1009,411){\makebox(0,0){$\ast$}}
\put(1071,326){\makebox(0,0){$\ast$}}
\put(1132,260){\makebox(0,0){$\ast$}}
\put(1193,219){\makebox(0,0){$\ast$}}
\put(1255,195){\makebox(0,0){$\ast$}}
\put(1316,184){\makebox(0,0){$\ast$}}
\put(1378,179){\makebox(0,0){$\ast$}}
\put(1439,178){\makebox(0,0){$\ast$}}
\put(1349,736){\makebox(0,0){$\ast$}}
\sbox{\plotpoint}{\rule[-0.500pt]{1.000pt}{1.000pt}}%
\sbox{\plotpoint}{\rule[-0.200pt]{0.400pt}{0.400pt}}%
\put(1279,695){\makebox(0,0)[r]{stark sauer}}
\sbox{\plotpoint}{\rule[-0.500pt]{1.000pt}{1.000pt}}%
\multiput(1299,695)(20.756,0.000){5}{\usebox{\plotpoint}}
\put(1399,695){\usebox{\plotpoint}}
\put(272,446){\usebox{\plotpoint}}
\multiput(272,446)(20.491,-3.305){4}{\usebox{\plotpoint}}
\multiput(334,436)(16.311,-12.835){3}{\usebox{\plotpoint}}
\multiput(395,388)(16.926,-12.012){4}{\usebox{\plotpoint}}
\multiput(457,344)(16.964,-11.958){4}{\usebox{\plotpoint}}
\multiput(518,301)(17.095,-11.770){3}{\usebox{\plotpoint}}
\multiput(579,259)(17.949,-10.422){4}{\usebox{\plotpoint}}
\multiput(641,223)(19.093,-8.138){3}{\usebox{\plotpoint}}
\multiput(702,197)(20.173,-4.881){3}{\usebox{\plotpoint}}
\multiput(764,182)(20.620,-2.366){3}{\usebox{\plotpoint}}
\multiput(825,175)(20.744,-0.680){3}{\usebox{\plotpoint}}
\multiput(886,173)(20.753,-0.335){3}{\usebox{\plotpoint}}
\multiput(948,172)(20.756,0.000){3}{\usebox{\plotpoint}}
\multiput(1009,172)(20.753,-0.335){3}{\usebox{\plotpoint}}
\multiput(1071,171)(20.753,0.340){3}{\usebox{\plotpoint}}
\multiput(1132,172)(20.756,0.000){3}{\usebox{\plotpoint}}
\multiput(1193,172)(20.753,0.335){2}{\usebox{\plotpoint}}
\multiput(1255,173)(20.753,0.340){3}{\usebox{\plotpoint}}
\multiput(1316,174)(20.753,0.335){3}{\usebox{\plotpoint}}
\multiput(1378,175)(20.753,0.340){3}{\usebox{\plotpoint}}
\put(1439,176){\usebox{\plotpoint}}
\put(272,446){\raisebox{-.8pt}{\makebox(0,0){$\Box$}}}
\put(334,436){\raisebox{-.8pt}{\makebox(0,0){$\Box$}}}
\put(395,388){\raisebox{-.8pt}{\makebox(0,0){$\Box$}}}
\put(457,344){\raisebox{-.8pt}{\makebox(0,0){$\Box$}}}
\put(518,301){\raisebox{-.8pt}{\makebox(0,0){$\Box$}}}
\put(579,259){\raisebox{-.8pt}{\makebox(0,0){$\Box$}}}
\put(641,223){\raisebox{-.8pt}{\makebox(0,0){$\Box$}}}
\put(702,197){\raisebox{-.8pt}{\makebox(0,0){$\Box$}}}
\put(764,182){\raisebox{-.8pt}{\makebox(0,0){$\Box$}}}
\put(825,175){\raisebox{-.8pt}{\makebox(0,0){$\Box$}}}
\put(886,173){\raisebox{-.8pt}{\makebox(0,0){$\Box$}}}
\put(948,172){\raisebox{-.8pt}{\makebox(0,0){$\Box$}}}
\put(1009,172){\raisebox{-.8pt}{\makebox(0,0){$\Box$}}}
\put(1071,171){\raisebox{-.8pt}{\makebox(0,0){$\Box$}}}
\put(1132,172){\raisebox{-.8pt}{\makebox(0,0){$\Box$}}}
\put(1193,172){\raisebox{-.8pt}{\makebox(0,0){$\Box$}}}
\put(1255,173){\raisebox{-.8pt}{\makebox(0,0){$\Box$}}}
\put(1316,174){\raisebox{-.8pt}{\makebox(0,0){$\Box$}}}
\put(1378,175){\raisebox{-.8pt}{\makebox(0,0){$\Box$}}}
\put(1439,176){\raisebox{-.8pt}{\makebox(0,0){$\Box$}}}
\put(1349,695){\raisebox{-.8pt}{\makebox(0,0){$\Box$}}}
\sbox{\plotpoint}{\rule[-0.200pt]{0.400pt}{0.400pt}}%
\put(211.0,131.0){\rule[-0.200pt]{0.400pt}{175.375pt}}
\put(211.0,131.0){\rule[-0.200pt]{295.825pt}{0.400pt}}
\put(1439.0,131.0){\rule[-0.200pt]{0.400pt}{175.375pt}}
\put(211.0,859.0){\rule[-0.200pt]{295.825pt}{0.400pt}}
\end{picture}

\end{figure}
Mittelt man nun jeweils die vier errechneten $K_p$, bzw $\ln{K_p}$ für eine Temperatur und plottet diese gemittelten $\ln{K_p}$ über die jeweils entsprechende reziproke Temperatur erhält man Abbildung~\ref{fig:plot} mit einer gefitteten y Kurve, die der Gleichung~\ref{eq:lnK2} mit Steigung $-\frac{\Delta_r H^0}{R} = 5219,6804$ entspricht. Daraus errechnet sich eine Reaktionsenthalpie $\Delta_r H^0 = -R\cdot 5219,6804= -43398,8229 \frac{\text{J}}{\text{mol}}$. Die Reaktionenthalpie zur Reaktionsgleichung \ch{N2 + 3 H2 <=> 2 NH3} beträgt nach Literaturangaben\footnote{\url{https://de.wikipedia.org/wiki/Haber-Bosch-Verfahren#Synthesebedingungen}} $\Delta H_{2\ch{NH3}} = -92,28$ kJ. Daraus folgt eine Reaktionsenthalpie $\Delta H_{\ch{NH3}} =\frac{1}{2} \Delta H_{2\ch{NH3}} = -46,14$ kJ für die Reaktion \ch{1/2 N2 + 3/2 H2 <=> NH3}. Die Abweichung von diesem Wert und dem gemessenen liegt bei etwa $5,94\%$.
\section{Literatur}
\begin{enumerate}[label=(\arabic*)]
	\item \emph{Praktische Einführung in die Chemie
für Studierende der Fachrichtungen
Technische Biologie und Physik}. Praktikumsskript, Universität Stuttgart,
SoSe 2017.  
	\item Prof. Dr. D. Gudat. \emph{„Einführung in die Chemie für Naturwissenschaftler“}. Vorlesungsskript
	\item \emph{Das Basiswissen der Chemie}. Charles E. Mortimer, Ulrich Müller. 12. Auflage
\end{enumerate}
\end{document}		
